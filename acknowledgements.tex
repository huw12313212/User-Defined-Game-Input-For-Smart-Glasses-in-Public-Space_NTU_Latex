\begin{acknowledgementszh}
首先誠摯的感謝指導教授陳彥仰博士,在研究所期間的悉心指導使我得以一窺人機互動領域的門徑。討論時循循善誘的引導方式,讓我在研究上獲益匪淺。在實驗室的這段時間老師以身作則的示範了許多世界級的做事方式是我將來學習的典範。

本論文的完成另外亦得感謝童盈超、許鈞彥、邱詩雅、林哲緯、吳沛容及林欣穎等人的大力協助,因為有你們的體諒與幫忙,才使得本論文能夠完整而嚴謹。

在台大人機互動實驗室的兩年,我有幸跟世界上最頂尖的一群天才創造了許多回憶,認真的學術討論、言不及意的鬼扯淡、深夜的大腸花、趕死線時的革命情感、垃圾清不掉的叫罵聲....這些是我們之間不可取代的羈絆。感謝眾位學長姐、同學、學弟妹的共同砥礪(墮落?),你/妳們的陪伴讓我兩年的研究生活變得多采多姿。

感謝許富傑、蔡銘綸、梁翔勝、蔡明璋、李孟翰、王正堯、王偉翰、陳厚任、鄭達陽、詹力韋、黃大源、梁容豪學長及簡妙恩、黃易學姊,你們的悉心指導與金玉良言,對我這兩年的研究生活影響深遠。也感谢李柏緯、何駿銘、張巧慧同學的幫忙,恭喜我們順利度過這兩年。實驗室的王權、周冠廷、修敏傑學弟、林敬沂學妹們當然也不能忘記,你/妳們聽我講垃圾話辛苦了。


女朋友王玉玲在背後默默支持更是我前進的動力,相信沒有妳的陪伴這兩年的生活會是很不一樣的光景。

最後再次感謝我在研究所的這段時間所遇到的所有人,是你們成就了這篇論文的每字每句。


\end{acknowledgementszh}

%\begin{acknowledgementsen}
%I'm glad to thank\ldots
%\end{acknowledgementsen}
