\begin{abstractzh}
智慧型眼鏡(如:Google Glass)與傳統的主機或行動遊戲平台不同,具備有隨時都可遊玩的特性,有創造無處不在的遊戲體驗的潛質。然而現行的智慧型眼鏡遊戲操作方式侷限在現有的軟硬體感應科技上。為了瞭解使用者真正想要的設計方向,在本篇論文中我們跳脫現行科技限制,探索了使用者在公眾場合中喜歡的智慧型眼鏡遊戲操作方式。

我們找了二十四名受測者執行了一場使用者定義遊戲操作實驗(User-Defined Game Input Study),實驗範圍囊括了十七種常見的遊戲操作、三種類別的人機互動方式跟兩種不同型式的智慧型眼鏡,最後共執行了兩千四百四十八次的遊戲操作嘗試。

根據我們的結果表示,相較於手持操作器,使用者有顯著性差異的較喜歡使用非觸碰式的操作(如:空中手勢)。在非手持的觸碰式操作中,使用者最喜歡的互動操作位置是手掌而不是穿戴式裝置(51\% vs 20\%)。
除此之外還發現在公眾場合中使用者會考慮社會認可的問題(Issue of Soical Acceptance) ,所以較喜歡使用不引人注意的操作方式,導致在使用空中手勢時使用者喜歡的操作範圍是在軀體(Torso)前方而非面前(63\% vs 37\%)。
\end{abstractzh}

\begin{abstracten}

Smart glasses, such as Google Glass, provide always-available displays not offered by console and mobile gaming devices, and could potentially offer a pervasive gaming experience. 
However, research on input for games on smart glasses has been constrained by the available sensors to date. 
To help inform design directions, this paper explores user-defined game input for smart glasses beyond the capabilities of current sensors, and focuses on the interaction in public settings. 
We conducted a user-defined input study with 24 participants, each performing 17 common game control tasks using 3 classes of interaction and 2 form factors of smart glasses, for a total of 2448 trials. 
Results show that users significantly preferred non-touch and non-handheld interaction over using handheld input devices, such as in-air gestures. Also, for touch input without handheld devices, users preferred interacting with their palms over wearable devices (51\% vs 20\%). 
In addition, users preferred interactions that are less noticeable due to concerns with social acceptance, and preferred in-air gestures in front of the torso rather than in front of the face (63\% vs 37\%). 

\end{abstracten}

\begin{comment}
\category{K.8.0.}{General}Games; {H.5.2.}{Information Interfaces}

\terms{Design, Human factors, Performance.}

\keywords{Game; input; Control; Smart glasses; Guessability; User-defined; Public space; Pervasive gaming; Wearable.}
\end{comment}

