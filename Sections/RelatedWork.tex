\chapter{Related Work}
\label{c:related work}

Related works fall within two areas: cooperative game design, and body language.

\section{Cooperative Game Design}

Several prior works have explored and analyzed cooperative game design patterns. Zagal et al. explored cooperative patterns within board games and yielded some observations that game designers might consider useful for designing collaborative game \cite{CG1}, and it also presented an ontology with a view to analyzing game play \cite{CG3}. 
Bjork and Holopainen presented a large quantity of game design patterns \cite{CG2}, including cooperative and social interaction patterns.
Rocha et al. presented a framework of several cooperative game design patterns and analyzed the actual impact of using these game mechanics to design a cooperative video game \cite{CG4}.
El-Nasr et al. extended Rocha et al.'s model and proposed Cooperative Performance Metrics (CPMs) to evaluate game experience \cite{CPMs}.

Wolmet et al. reported a study of how parents and children play several cooperative co-located games with different characteristics \cite{CG5}. 
Mark et al. \cite{CG6} evaluated the communicative and cooperative behavior of same-age and mixed-age pairs (Young-Young, Young-Old, Old-Old), and identified noticeable difference between group types. 
Hamilton et al. \cite{CG7} explored how to design games for children to play with cerebral palsy, and it also presented several cooperative gameplay prototypes.

In our work, we explored and evaluated the possibility to use body language as a communication manner in cooperative game design, and analyzed the communication pattern with players.

\section{Body Language}

Consist of human communication, there is not only speech but also inclusive of various gestures and body motions. Body language, a non-verbal way to transmit your thoughts without verbalizing. According to The 7\% Rule\cite{GD2}, the influence of communication for verbal is only 7\% but is 93\% for non-verbal expression. And the non-varbal expression is made up of body language (55\%) and tones of voice (38\%).

Charades\cite{GD3} is a word guessing game. It is an acting game in which one player act as a word or a phrase, and sometimes imitates a similar pronounced words, while the other players guess the answer. The main idea is to use the body to make physical expression rather than using verbal language. 

Inspired by The 7\% Rule and the Charades, we suggested using body language as a communication manner in cooperative game to normalize player's communication skill. With this idea, whether players are playing with different language speakers or not, their communication skill is near enough for a game developer to design a proper difficulty to entertain players. On the other hand, many researchers have argued that the body movement brings about a positive emotional and social response \cite{GD7, GD8, GD9}. We believe that body language communication should enhance game engagement and enjoyment.
